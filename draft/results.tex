\section{Results}
\label{sec:results}
In this section we present numerical results for the NLO EW 
corrections to the production of a diphoton pair in 
association with two bottom jets at the LHC at a centre-of-mass energy of 13\,TeV. 
To assess the importance of the bottom quark mass we calculated this process for both
massless and massive $b$-quarks.
All results are obtained in the Standard Model using the complex-mass 
scheme \cite{Denner:2005fg} with the following input parameters
\begin{center}
  \begin{tabular}{rclrcl}
    $\alphazero$ &\shortequal& $1/137.03599976$  \qquad &&& \\
%     $\Gmu$ &\shortequal& $1.1663787\times 10^{-5}\; \text{GeV}^2$ &&& \\
    $m_W$ &\shortequal& $80.385\; \text{GeV}$       & $\Gamma_W$ &\shortequal& $2.085\; \text{GeV}$ \\
    $m_Z$ &\shortequal& $91.1876\; \text{GeV}$      & $\Gamma_Z$ &\shortequal& $2.4952\; \text{GeV}$ \\
    $m_h$ &\shortequal& $125.0\; \text{GeV}$        & $\Gamma_h$ &\shortequal& $0$\\
    $m_t$ &\shortequal& $173.2\; \text{GeV}$        & $\Gamma_t$ &\shortequal& $0$\;.
  \end{tabular}
\end{center}
In the case of a massive $b$-quark, its mass has been set to
\begin{equation}
m_b = 4.75 \; \text{GeV}.
\end{equation}
All other lepton and parton masses and widths are set to zero. 
The width of the top quark can safely be neglected as there 
are no diagrams with a $s$-channel resonance that can go on-shell. 
At the same time, while Higgs-boson contributions are absent in 
the calculation with a massless $b$ quark, they enter in the 
presence of a $b$ quark mass.

Events are selected by requiring the presence of at least two 
isolated photons defined through the smooth cone isolation 
criterion \cite{Frixione:1998jh}. 
This isolation criterion puts an upper limit on the hadronic 
actiivity in a cone of size $R_\gamma$ around a photon
\begin{equation}
  \begin{split}
    E_\text{had,max}(r_\gamma)
    \,=\;&
      \epsilon\,\pT^\gamma
      \left(\frac{1-\cos r_\gamma}{1-\cos R_\gamma}\right)^n\;.
  \end{split}
\end{equation}
Therein, $r_\gamma$ denotes the angular separation between the photon 
and the parton, and
\begin{equation}
  \begin{split}
    R_\gamma=0.4\;,
    \qquad
    \epsilon=0.05\;,
    \qquad
    n=1
  \end{split}
\end{equation}
are free parameters. 
Of those, at least two photons are required to lie in the fiducial 
volume given by 
\begin{equation}
  \begin{split}
    \pT^\gamma > 30\,\text{GeV}\;,
    \qquad
    |\eta_\gamma| < 2.5\;.
  \end{split}
\end{equation}
Please note, that in NLO EW calculations more than two isolated photons 
may be found fulfilling the above criteria.
The two isolated photons in the fiducial volume with the largest 
transverse momenta will be subsequently referred to as leading and 
subleading photon, $\gamma_1$ and $\gamma_2$.
All remaining partons and non-isolated photons are used to build 
jets with the anti-$k_t$ \cite{Cacciari:2008gp} as implemented 
in \textsc{FastJet} \cite{Cacciari:2011ma} with $R=0.4$. 
We require the presence of at least two $b$ jets with 
\begin{equation}
  \begin{split}
    \pT^j > 20\,\text{GeV}\;,
    \qquad
    |\eta_j| < 4.4\;.
  \end{split}
\end{equation}
In the calculation using massless $b$ quarks, jets containing an equal 
number of $b$ and $\bar{b}$ quarks are not tagged as $b$ jets.
\comment{MS: Clarify: $b$-jets out to 4.4?}

We define our central scales through
\begin{equation}
  \label{eq:murfdef}
    \muR^0 = \muF^0 = \frac{1}{2}\sqrt{m^2_{\gamma \gamma} +\left( \sum_i p_{T,i}\right)^2}\;,
\end{equation}
where the sum runs over all final state partons and non-identified photons 
while $m_{\gamma\gamma}$ is defined as the invariant mass of the 
leading and subleading isolated photon defined above.
To estimate the uncertainty originating in yet-to-be-calculated 
higher order contributions we use the scale-dependence of the 
differential cross section as a proxy, varying both the 
renormalisation and factorisation 
independently by the conventional factor of two (excluding points 
where their ratio is larger than two) to form an uncertainy envelope.
In the case where the $b$-quark is massless
we use the \textsc{CT14nlo} PDF set with $\alphas(m_Z)=0.118$
\cite{Dulat:2015mca}, interfaced through \textsc{Lhapdf6} \cite{Buckley:2014ana}. 
For the massive case we employed the  4-flavor PDF set \textsc{CT14nlo\_NF4} with $\alphas(m_Z)=0.113$
In principle at NLO EW one also has to take photon induced processes 
into account. 
Their LO equivalents are negligible, however, justifying the use of 
pure QCD PDF sets, cf.\ Sec.\ \ref{sec:results:subLO-aind}.


All distributions have been analysed using \textsc{Rivet} \cite{Buckley:2010ar}.

\subsection{Fiducial cross sections}
\label{sec:results:xsecs}

\subsection{Differential distributions}
\label{sec:results:dists}

\subsection{Subleading Born contributions and photon induced processes}
\label{sec:results:subLO-aind}

