\section{Results}
\label{sec:results}
In this section we present numerical results for the NLO EW 
corrections to the production of a diphoton pair in 
association with two bottom jets at the LHC at a centre-of-mass energy of 13\,TeV. 
To assess the importance of the bottom quark mass we calculated this process for both
massless and massive $b$-quarks.
All results are obtained in the Standard Model using the complex-mass 
scheme \cite{Denner:2005fg} with the following input parameters
\begin{center}
  \begin{tabular}{rclrcl}
    $\alphazero$ &\shortequal& $1/137.03599976$  \qquad &&& \\
    $\Gmu$ &\shortequal& $1.1663787\times 10^{-5}\; \text{GeV}^2$ &&& \\
    $m_W$ &\shortequal& $80.385\; \text{GeV}$       & $\Gamma_W$ &\shortequal& $2.085\; \text{GeV}$ \\
    $m_Z$ &\shortequal& $91.1876\; \text{GeV}$      & $\Gamma_Z$ &\shortequal& $2.4952\; \text{GeV}$ \\
    $m_h$ &\shortequal& $125.0\; \text{GeV}$        & $\Gamma_h$ &\shortequal& $0$\\
    $m_t$ &\shortequal& $173.2\; \text{GeV}$        & $\Gamma_t$ &\shortequal& $0$\;.
  \end{tabular}
\end{center}
In the case of a massive $b$-quark, its mass has been set to
\begin{equation}
m_b = 4.75 \; \text{GeV}.
\end{equation}
All other lepton and parton masses and widths are set to zero. 
In the case where the $b$-quark is massless
we use the \textsc{CT14nlo} PDF set with $\alphas(m_Z)=0.118$
\cite{Dulat:2015mca}, interfaced through LHAPDF6 \cite{Buckley:2014ana}. 
For the massive case we employed the  4-flavor PDF set \textsc{CT14nlo\_NF4} with $\alphas(m_Z)=0.113$
In principle one also has to take photon induced processes into account, however as we will show
later in this next section, their contributions are negligible which justifies the use of pure QCD pdf sets.\\
We define our central scales through
\begin{equation}
  \label{eq:murfdef}
    \muR^0 = \muF^0 = \frac{1}{2}\sqrt{m^2_{\gamma \gamma} +\left( \sum_i p_{T,i}\right)^2}\;,
\end{equation}
where the sum runs over the final state partons.