\section{Introduction}
\label{sec:intro}

One of the main goals of a high luminosity extension of the LHC will be to provide a more precise insight into the Higgs self 
coupling. Although the Higgs boson found at the LHC \cite{Aad:2012tfa,Chatrchyan:2012xdj} is in good agreement with the
Standard Model prediction for the processes that are feasible so far, the current measurements cannot give a final answer
yet whether the Higgs bosons is a Standard Model Higgs or not. Various BSM scenarios can make very similar predictions
concerning coupling strength, branching ratios etc. Further insight can be given by the measurement of the Higgs self coupling
which can be assessed by the double Higgs production channel. For the given value of the Higgs mass several decay channels
seem to be accessible and have been studied by both ATLAS and CMS, such as $\gamma\gamma
b\bar{b}$~\cite{TheATLAScollaboration:2016ibb,Khachatryan:2016sey,Aad:2015xja,Aad:2014yja},
$b\bar{b}b\bar{b}$~\cite{Aaboud:2016xco,CMS:2016tlj,Aad:2015xja,Khachatryan:2015yea,Aad:2015uka},
$\gamma\gamma W W^*$, $b\bar{b}W W^*$,
$\tau^+\tau^-b\bar{b}$~\cite{ATLAS:2016qmt,CMS:2016cdj,CMS:2016ymn,CMS:2016rec,CMS:2016guv,CMS:2016ugf,CMS:2016zxv,Aad:2015xja}.

The dominant production mechanism for double Higgs production is the gluon fusion channel, where either both Higgs bosons
are radiated off a heavy fermion loop, or where one off-shell Higgs is radiated off the fermion loop and decays into two Higgs
bosons. Only the latter of the two involves the Higgs self coupling.  For single Higgs production one can obtain reliable results
at least for the total cross section by the use of the effective theory where the top quark is treated as being infinitely heavy.
This works as the top quark is heavier than the Higgs. However having two Higgs particles, they are resolving the top-quark
loop and the effective theory breaks down. Therefore reliable results for double Higgs production can only be obtained  within
the full theory. In the full theory the process is known at leading order \cite{Eboli:1987dy,Glover:1987nx,Plehn:1996wb}, and in various approximations
taking higher order corrections into account \cite{Dawson:1998py,Maltoni:2014eza,Grigo:2013rya,Grigo:2014jma,Grigo:2015dia,Degrassi:2016vss,
deFlorian:2013uza,deFlorian:2013jea,Shao:2013bz,deFlorian:2015moa,deFlorian:2016uhr}. Only very recently the NLO result taking full top
mass dependence into account became available \cite{Borowka:2016ehy,Borowka:2016ypz}.

The process where one of the Higgs decays into a $b\bar{b}$ pair and the other Higgs into a pair of photons can be seen 
as a compromise, as the first one benefits from a large branching ratio but suffers from large QCD backgrounds, whereas the 
second suffers from a small branching ratio but benefits from a clean signal that is relatively easy to detect. A measurement
of this process requires the precise knowledge of the corresponding background process. The background process 
$pp\to \gamma \gamma b \bar{b}$ is known at NLO in QCD \cite{Faeh:2017fpp}. 

In this paper we improve on the existing 
results by calculating the next-to-leading
order electroweak corrections to this process for both massless and massive $b$-quarks. The paper is organized as follows.
In section \ref{sec:setup} we described the calculational setup that we have used to obtain the numerical results before we
discuss the results in section \ref{sec:results}. Finally, we conclude in section \ref{sec:conclusions}.