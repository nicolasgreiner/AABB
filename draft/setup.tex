\section{Calculational setup}
\label{sec:setup}

The results presented in this paper have been obtained by combining the two tools
\GoSam~\cite{Cullen:2011ac,Cullen:2014yla} and \Sherpa~\cite{Gleisberg:2008ta}
allowing for a fully automated calculation of cross section and observables and 
next-to-leading order in QCD as well as in the electroweak coupling.
\GoSam is a package which generates the code for the numerical evaluation of
the one loop scattering amplitudes starting from the Feynman diagrams,
generated with \QGraf~\cite{Nogueira:1991ex} and further processed with
\FORM~\cite{Vermaseren:2000nd,Kuipers:2012rf} and
\Spinney~\cite{Cullen:2010jv} to perform necessary algebraic
manipulations to obtain an optimized expression for the matrix elements.
For the integrand reduction of the diagrams we use the \Ninja
library~\cite{Peraro:2014cba}, an implementation of the technique of integrand
reduction via Laurent expansion~\cite{Mastrolia:2012bu,vanDeurzen:2013saa}.
Alternatively one can choose other reduction strategies such as OPP reduction
method~\cite{Ossola:2006us,Mastrolia:2008jb,Ossola:2008xq} which is
implemented in $d$ dimensions in \Samurai~\cite{Mastrolia:2010nb}, or methods based on
tensor integral reduction as implemented in
\GolemNF~\cite{Heinrich:2010ax,Binoth:2008uq,Cullen:2011kv,Guillet:2013msa}.
We have used \OneLoop~\cite{vanHameren:2010cp} to evaluate the scalar integrals.

The tree-level matrix elements as well as the infrared subtraction, 
process management and phase-space integration of all contributing 
partonic channels, on the other hand, are provided by \Sherpa 
through its tree-level matrix element generator 
\textsc{Amegic} \cite{Krauss:2001iv}. 
The subtraction of infrared QCD and QED divergences for both the 
calculation with massless and massive $b$-quarks are performed using a 
generalisation of the Catani-Seymour scheme~\cite{Catani:1996vz,
  Dittmaier:1999mb,Catani:2002hc,Gleisberg:2007md,Archibald:2011nca,
  Kallweit:2014xda,Kallweit:2015dum,Kallweit:2017khh,Schonherr:2017},
the appropriate initial state mass factorisation counter terms 
are included.
Both programs, \Sherpa and \GoSam, are interfaced through a 
dedicated interface based on the 
Binoth Les Houches Accord~\cite{Binoth:2010xt,Alioli:2013nda}. 
Cross-checks of the tree-level matrix elements of \GoSam and 
\Sherpa and the renormalized pole coefficients of the virtual 
corrections of \GoSam and the infrared poles of \Sherpa have 
been performed for several phase space points spanning 
multiple kinematic regimes and excellent agreement has been found.

